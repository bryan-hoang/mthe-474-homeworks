% Provides macros manipulating strings of tokens.
\RequirePackage{xstring}

% Store the jobname as a string with category 11 characters.
\edef\normaljobname{\expandafter\scantokens\expandafter{\jobname\noexpand}}
\StrBetween{\normaljobname}{hw-}{-q}[\homeworknumber]
\StrBehind{\normaljobname}{-q-}[\questionnumber]

\documentclass[
  coursecode={MTHE 474},
  assignmentname={Homework \homeworknumber},
  studentnumber=20053722,
  name={Bryan Hoang},
  draft,
  % final,
]{
  ltxanswer%
}

\usepackage{bch-style}

\makeatletter
\def\pgfmathreciprocal@#1{%
    \begingroup
    % FIXME optimize
    \pgfkeys{/pgf/fpu=true,/pgf/fpu/output format=fixed}%
    \pgfmathparse{1/#1}%
    \pgfmath@smuggleone\pgfmathresult
    \endgroup
}%
\makeatother

\begin{document}
  \begin{questions}
    \setcounter{question}{\questionnumber}
    \addtocounter{question}{-1}
    \question[15]\
    \begin{parts}
      \part{}
      \begin{solution}
        Denote \(f(D)=\sum_{i=1}^{7} D^{-l_{i}}\) For the integers \(l_{1},\dotsc,l_{7}\) to satisfy Kraft's inequality, we want
        \begin{align*}
          f(D) &= \sum_{i=1}^{7} D^{-l_{i}}                                                                \\
               &= D^{-l_{1}} + D^{-l_{2}} + D^{-l_{3}} + D^{-l_{4}} + D^{-l_{5}} + D^{-l_{6}} + D^{-l_{7}} \\
               &= 2D^{-1} + 2D^{-2} + 2D^{-3} + D^{-4}                                                     \\
               &\le 1
        \end{align*}
        Trying values of \(D \in \Z_{\ge2}\) yields
        \begin{align*}
          f(2) &= \frac{29}{16} \nleq 1 \\
          f(3) &= \frac{79}{81} \leq 1.
        \end{align*}
        \(\therefore\) the integers \(l_{1},\dotsc,l_{7}\) satisfy Kraft's inequality in base \(\boxed{D=3}\).
      \end{solution}

      \part{}
      \begin{solution}
        \begin{claim}
          \(\boxed{l_{7}^{*}=3}\) makes \(l_{1},\dotsc,l_{6},l_{7}^{*}\) satisfy Kraft's inequality in base 3 with equality.
        \end{claim}
        \begin{proof}
          \begin{align*}
            f(3) &= 3^{-l_{1}} + 3^{-l_{2}} + 3^{-l_{3}} + 3^{-l_{4}} + 3^{-l_{5}} + 3^{-l_{6}} + 3^{-l_{7}^{*}} \\
                 &= 2(3)^{-1} + 2(3)^{-2} + 3(3)^{-3}                                                            \\
                 &= 1
          \end{align*}
        \end{proof}
      \end{solution}

      \newpage
      \part{}
      \begin{solution}
        \begin{answerfigure}
          \begin{forest} huffman tree
              [Root node,plain content
                  [c_{1}]
                  [c_{2}]
                  [
                    [c_{3}]
                    [c_{4}]
                    [
                      [c_{5}]
                      [c_{6}]
                      [c_{7}]
                    ]
                  ]
              ]
          \end{forest}
          \caption{Ternary Huffman tree}
          \label{fig:huffman-tree}
        \end{answerfigure}
        From Figure~\ref{fig:huffman-tree}, a ternary prefix code with the integers \(l_{1},\dotsc,l_{6},l_{7}^{*}\) as its codeword lengths is \(\boxed{\mathcal{C}=\{0,1,20,21,220,221,222\}}\).
      \end{solution}

      \part{}
      \begin{solution}
        Let \(\{X_{i}\}_{i=1}^{\infty}\) be a DMS with alphabet \(\mathcal{X}=\{a_{1},\dotsc,a_{7}\}\) and PMF \(p_{X}(a_{i})=p_{i}=3^{-l_{i}}\) where \(l_{i}\) is the length of codeword \(c_{i}\) for source symbol \(a_{i}\) for \(i\in\{1,\dotsc,7\}\). Then the Ternary Huffman code is
        \begin{align*}
          f:\mathcal{X} &\to \{0,1,2\}^{*} \\
          a_{1}         &\to 0             \\
          a_{2}         &\to 1             \\
          a_{3}         &\to 20            \\
          a_{4}         &\to 21            \\
          a_{5}         &\to 220           \\
          a_{6}         &\to 221           \\
          a_{7}         &\to 222.
        \end{align*}
        The source's entropy is
        \begin{align*}
          H_{3}(X)     &= -\sum_{i=1}^{7} p_{i} \log_3 p_{i}                 \\
          \alignedbox{ &= \qty[parse-numbers=false]{1.\overline{4}}{\trits}}
        \end{align*}
      \end{solution}
    \end{parts}
  \end{questions}
\end{document}
