% Provides macros manipulating strings of tokens.
\RequirePackage{xstring}

% Store the jobname as a string with category 11 characters.
\edef\normaljobname{\expandafter\scantokens\expandafter{\jobname\noexpand}}
\StrBetween{\normaljobname}{hw-}{-q}[\homeworknumber]
\StrBehind{\normaljobname}{-q-}[\questionnumber]

\documentclass[
  coursecode={MTHE 474},
  assignmentname={Homework \homeworknumber},
  studentnumber=20053722,
  name={Bryan Hoang},
  draft,
  % final,
]{
  ltxanswer%
}

\usepackage{bch-style}

\begin{document}
  \begin{questions}
    \setcounter{question}{\questionnumber}
    \addtocounter{question}{-1}
    \question[10]\
    \begin{parts}
      \part{}
      \begin{solution}
        \(C=\{01,010,101\}\) is \textbf{not UD} since, for example, suppose we have that
        \begin{align*}
          a_{1} &\to 10   \\
          a_{2} &\to 010  \\
          a_{3} &\to 101.
        \end{align*}
        Then
        \begin{equation*}
          101010
        \end{equation*}
        can be decoded into codewords \(\underbrace{10}_{a_{1}},\underbrace{10}_{a_{1}},\underbrace{10}_{a_{1}}\) or codewords \(\underbrace{101}_{a_{3}},\underbrace{010}_{a_{2}}\).
      \end{solution}

      \part{}
      \begin{solution}
        \(C=\{0,01,011,0111\}\) is \textbf{UD} since it is a suffix code.
      \end{solution}

      \part{}
      \begin{solution}
        \(C=\{21,20,201,202,212\}\) is \textbf{UD} since it is a suffix code.
      \end{solution}

      \part{}
      \begin{solution}
        \(C=\{1,21,221,002,021,001\}\) is \textbf{UD} since it is a prefix code.
      \end{solution}

      \part{}
      \begin{solution}
        \(C=\{10,12,13,22,121,133,220,221,223\}\) is \textbf{UD} since it is a suffix code.
      \end{solution}
    \end{parts}
  \end{questions}
\end{document}
