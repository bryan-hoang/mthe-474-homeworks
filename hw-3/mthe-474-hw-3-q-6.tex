% Provides macros manipulating strings of tokens.
\RequirePackage{xstring}

% Store the jobname as a string with category 11 characters.
\edef\normaljobname{\expandafter\scantokens\expandafter{\jobname\noexpand}}
\StrBetween{\normaljobname}{hw-}{-q}[\homeworknumber]
\StrBehind{\normaljobname}{-q-}[\questionnumber]

\documentclass[
  coursecode={MTHE 474},
  assignmentname={Homework \homeworknumber},
  studentnumber=20053722,
  name={Bryan Hoang},
  draft,
  % final,
]{
  ltxanswer%
}

\usepackage{bch-style}

\begin{document}
  \begin{questions}
    \setcounter{question}{\questionnumber}
    \addtocounter{question}{-1}
    \question[15]\
    \begin{parts}
      \part{}
      \begin{solution}
        Let \(\mathcal{S}\) and \(\mathcal{U}\) denote the set of all suffix and UD first-order binary VLCs, respectively.

        First, note that since \(\mathcal{S}\subseteq\mathcal{U}\), then
        \begin{align*}
          \min_{\C\in\mathcal{U}}\bar{L}  &\le \min_{\C\in\mathcal{S}}\bar{L}                         \\
          \Rightarrow \bar{L}_{\text{UD}} &\le \bar{L}_{\text{S}}\numberthis\label{eq:ud-less-than-s}
        \end{align*}
        Now let \(\C^{*}\), with codeword lengths \(l_{1}^{*},\dotsc,l_{n}^{*}\), be the \underline{optimal} UD code \((\C^{*}\in\mathcal{U})\):
        \begin{equation*}
          \bar{L}(\C^{*}) = \bar{L}_{\text{UD}}
        \end{equation*}
        where \(\bar{L}(\C)\) is the expected cost function, \(\bar{L}\), of \(\C\).

        Then the codeword lengths \(\{l_{i}^{*}\}_{i=1}^{M}\) of \(\C^{*}\) must satisfy the Kraft inequality (in base D) since \(\C^{*}\) is UD.

        Then by the Theorem in class on the ``Kraft Inequality for Prefix codes'', \(\exists\) prefix code \(\C^{'}\) with the same codeword lengths as \(C^{*}\): \(l_{i}^{'}=l_{i}^{*},\ i=1,\dotsc,M\). We can reverse the codewords of \(\C^{'}\) to obtain a suffix code \(\C^{''}\in\mathcal{S}\), with the same codeword lengths as \(C^{*}\): \(l_{i}^{''}=l_{i}^{*},\ i=1,\dotsc,M\).
        \begin{align*}
          \therefore \bar{L}(\C^{''}) &= \sum_{i=1}^{M} p_{i}c(l_{i}^{''}) \\
                                      &= \sum_{i=1}^{M} p_{i}c(l_{i}^{*})  \\
                                      &= \bar{L}_{\text{UD}}
        \end{align*}
        \underline{But}, \(\bar{L}(\C^{''})\ge\bar{L}_{\text{S}}\) by the definition of \(\bar{L}_{\text{S}}\).
        \begin{equation}
          \therefore \bar{L}_{\text{UD}} \ge \bar{L}_{\text{S}}\label{eq:s-less-than-ud}.
        \end{equation}
        Then by~\eqref{eq:ud-less-than-s} and~\eqref{eq:s-less-than-ud}, we have \(\boxed{\bar{L}_{\text{UD}}=\bar{L}_{\text{S}}}\).
      \end{solution}

      \part{}
      \begin{solution}
        \underline{Compression efficiency} is one appropriate metric, where we want to choose a code alphabet size \(D\) and maximize order \(n\) so that the code's average code rate, \(\overline{R}\), is as close as possible to the DMSs source entropy, \(H_{D}(X)\).

        \underline{Encoding delay} is another metric, where we want to minimize order \(n\) and design a code with minimal variance in length to minimize the delay.

        \underline{Storage and computational complexity} is a third metric, where we want to minimize the size of the source alphabet by minimizing the order \(n\).

        Due to these considerations, we need to balance the metrics and evluate the trade-offs between different solutions depending on the design application.
      \end{solution}
    \end{parts}
  \end{questions}
\end{document}
