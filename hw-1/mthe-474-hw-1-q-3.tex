% Provides macros manipulating strings of tokens.
\RequirePackage{xstring}

% Store the jobname as a string with category 11 characters.
\edef\normaljobname{\expandafter\scantokens\expandafter{\jobname\noexpand}}
\StrBetween{\normaljobname}{hw-}{-q}[\homeworknumber]
\StrBehind{\normaljobname}{-q-}[\questionnumber]

\documentclass[
  coursecode={MTHE 474},
  assignmentname={Homework \homeworknumber},
  studentnumber=20053722,
  name={Bryan Hoang}
]{
  ltxanswer%
}

\usepackage{bch-style}

\begin{document}
  \begin{questions}
    \setcounter{question}{\questionnumber}
    \addtocounter{question}{-1}
    \question{}
    \begin{parts}
      \
      \part{}
      \begin{solution}
        Since conditioning decreases entropy, we have that
        \begin{align*}
          H(X|Y)  &\ge H(X|Y,Z)                                  \\
          -H(X|Z) &\le -H(X|Y,Z)\numberthis\label{eq:inequality}
        \end{align*}
        We also have the following identity for the conditional mutual information:
        \begin{equation}\label{eq:identity}
          I(X;Y|Z) = H(X|Z) - H(X|Y,Z)
        \end{equation}
        Thus,
        \begin{align*}
          H(X|Z) - H(X|Y)             &\le H(X|Z) - H(X|Y,Z) & &\text{by~\eqref{eq:inequality}} \\
          \alignedbox{H(X|Z) - H(X|Y) &\le I(X;Y|Z)}         & &\text{by~\eqref{eq:identity}}
        \end{align*}
      \end{solution}

      \part{}
      \begin{solution}
        If \(Z = g(Y)\), then \(X \rightarrow Y \rightarrow Z\). Thus,
        \begin{align*}
          I(X;Y)                 &\ge I(X;g(Y))                           \\
          \Rightarrow -I(X;g(Y)) &\ge -I(X;Y)\numberthis\label{eq:mutual}
        \end{align*}
        By the properties of mutual information, we also have that
        \begin{align*}
          H(X|g(Y)) &= H(X) - I(X;g(Y)) &\text{and} & &H(X|Y) &= H(X) - I(X;Y)
        \end{align*}
        Then by~\eqref{eq:mutual}, it follows that
        \begin{align*}
          H(X) - I(X;g(Y))                  &\ge H(X) - I(X;Y) \\
          \Rightarrow \alignedbox{H(X|g(Y)) &\ge H(X|Y)}
        \end{align*}
      \end{solution}

      \part{}
      \begin{solution}
        \begin{align*}
          \frac{1}{2}H(X_{1},X_{2}) &= \frac{1}{2}(H(X_{2}|X_{1}) + H(X_{1})) & &\text{by the chain rule}                                              \\
                                    &= \frac{1}{2}(H(X_{2}|X_{1}) + H(X_{2})) & &\because X_{1} \text{ and } X_{2} \text{ are identically distributed} \\
                                    &\ge \frac{1}{2}(2 \cdot H(X_{2}|X_{1}))  & &\because \text{conditioning reduces entropy}                          \\
                                    &= H(X_{2}|X_{1})
        \end{align*}
        Therefore, we have \(\boxed{H(X_{2}|X_{1}) \le \frac{1}{2}H(X_{1},X_{2})}\).
      \end{solution}
    \end{parts}
  \end{questions}
\end{document}
