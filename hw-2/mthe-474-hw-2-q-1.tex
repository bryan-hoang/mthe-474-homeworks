% Provides macros manipulating strings of tokens.
\RequirePackage{xstring}

% Store the jobname as a string with category 11 characters.
\edef\normaljobname{\expandafter\scantokens\expandafter{\jobname\noexpand}}
\StrBetween{\normaljobname}{hw-}{-q}[\homeworknumber]
\StrBehind{\normaljobname}{-q-}[\questionnumber]

\documentclass[
  coursecode={MTHE 474},
  assignmentname={Homework \homeworknumber},
  studentnumber=20053722,
  name={Bryan Hoang},
  draft,
  % final,
]{
  ltxanswer%
}

\usepackage{bch-style}

\begin{document}
  \begin{questions}
    \setcounter{question}{\questionnumber}
    \addtocounter{question}{-1}
    \question[10]\
    \begin{parts}
      \part{}
      \begin{solution}
        \begin{example}
          Let the sample space be \(\mathcal{X}=\{0,1\}\) and consider the following three probability distributions
          \begin{align*}
            p(x) &= \begin{cases}
                      \frac{1}{2} &\text{if } x=0, \\
                      \frac{1}{2} &\text{if } x=1,
                    \end{cases} \\
            r(x) &= \begin{cases}
                      \frac{1}{4} &\text{if } x=0, \\
                      \frac{3}{4} &\text{if } x=1,
                    \end{cases} \\
            q(x) &= \begin{cases}
                      \frac{1}{8} &\text{if } x=0, \\
                      \frac{7}{8} &\text{if } x=1,
                    \end{cases}
          \end{align*}
          Then the triangle inequality for divergence implies that
          \begin{align*}
            D(p||q)                                                                                           &\le D(p||r) + D(r||q)                                                                                                                                                                                     \\
            \sum_{x\in\{0,1\}} p(x)\log\frac{p(x)}{q(x)}                                                      &\le \sum_{x\in\{0,1\}} p(x)\log\frac{p(x)}{r(x)} + \sum_{x\in\{0,1\}} r(x)\log\frac{r(x)}{q(x)}                                                                                                           \\
            \frac{1}{2} \log\frac{\frac{1}{2}}{\frac{1}{8}} + \frac{1}{2} \log\frac{\frac{1}{2}}{\frac{7}{8}} &\le \frac{1}{2} \log\frac{\frac{1}{2}}{\frac{1}{4}} + \frac{1}{2} \log\frac{\frac{1}{2}}{\frac{3}{4}} + \frac{1}{4} \log\frac{\frac{1}{4}}{\frac{1}{8}} + \frac{3}{4} \log\frac{\frac{3}{4}}{\frac{7}{8}} \\
            0.41                                                                                              &\le 0.20
          \end{align*}
          which is false, thus making Divergence violate the triangle inequality.
        \end{example}
      \end{solution}

      \part{}
      \begin{solution}
        \begin{proof}
          Let \(p\) and \(q\) be two PMFs on the same alphabet \(\mathcal{X}\).

          To apply Jensen's Inequality, let
          \begin{equation*}
            f(x)=-\log x
          \end{equation*}
          be a convex function and define the random variable \(Y\) to have the alphabet
          \begin{equation*}
            \mathcal{Y}=\left\{\frac{q(x)}{p(x)} : x \in \mathcal{X}\right\}
          \end{equation*}
          with the PMF
          \begin{equation*}
            P\left(Y=\frac{q(x)}{p(x)}\right)=p(x),\ \forall x \in \mathcal{X}
          \end{equation*}
          Then
          \begin{align*}
            D(p||q) &= \E_{p}\left[\log\frac{p(X)}{q(X)}\right]                                                                    \\
                    &= \E_{p}\left[-\log\frac{q(X)}{p(X)}\right]                                                                   \\
                    &= \E_{p}[f(Y)]                                                                                                \\
                    &\ge f(\E[Y])                                                                 & &\text{by Jensen's Inequality} \\
                    &= -\log\sum_{x\in\mathcal{X}} \frac{q(x)}{\cancel{p(x)}} \cdot \cancel{p(x)}                                  \\
                    &= -\log 1                                                                                                     \\
                    &= 0
          \end{align*}
          Therefore, divergence is non-negative.
        \end{proof}
      \end{solution}
    \end{parts}
  \end{questions}
\end{document}
